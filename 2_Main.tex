% \documentclass{article}
% \usepackage[utf8]{inputenc}

% \title{Mechanical Design Guide for Motion Applications}
% \author{marco.schneider }
% \date{April 2021}

% \begin{document}

% \maketitle

\section{Introduction}

\title{Device Motion Components, ElNet Motion}
\author{Marco Schneider, Niko Kivel}
\date{April 2021}



\maketitle

\section{Introduction}

Motion Design Guide
Marco Schneider

\section{Grundlagen}

\subsection{Umgebungsbedingung}
ESS Dokument für infos lesen

\subsection{Personensicherheit}


\subsection{Genauigkeit und Präzision}
Anforderungen klären

\subsection{Fahrweg}
Genügend Spielraum für Bremswege (nach Limitswitch)

\subsection{Geschwindigkeiten}
10 mm/s

\subsection{Endanschläge}
Genügend Stabil

\subsection{Motoren}

\subsection{Encoder}

\subsection{Limitswitches}
Immer zwei, jeweils für pos und neg richtung
Müssen überfahren werden könne

\subsection{Kabel und Stecker}

\subsection{Kupplungen}

\subsection{Linearführungen}

\subsection{Spindellagerung}

\subsection{Schleppketten}

\subsection{Berührungsschutz}
IP2x
	
\subsection{Weiteres}


% \end{document}

