\documentclass[english]{psi_easy}

%% Language and font encodings
\usepackage[english]{babel}
\usepackage[utf8x]{inputenc}
\usepackage[T1]{fontenc}
\usepackage{comment} 

%% Useful packages
\usepackage{amsmath}
\usepackage{graphicx}
\usepackage[table,xcdraw]{xcolor}
\usepackage{booktabs}
% \usepackage[colorinlistoftodos]{todonotes}
\usepackage[colorlinks=true, allcolors=blue]{hyperref}

% float placement
\usepackage{float}

% cable color bullets
\usepackage{tikz}

% for trademark sign 'circledR'
\usepackage{amssymb}

% for PSI logo on Rolfs title page
\usepackage{epsfig}

%% Set title here for EASY title page
\PsiEasyTitle{Guideline: Mechanical Design for Motion Applications}

% ---- custom commands
% Schneidersche Einrueckung in 'itemize'
\newcommand{\itemIndent}[2]{{\makebox[3cm]{#1:\hfill} #2.}}
% draw colored circle
\newcommand{\wireColor}[1]{\tikz\draw[black,fill=#1] (0,0) circle (.9ex); #1}
\newcommand{\EtherCAT}[0]{EtherCAT$^{\tiny{\circledR}}$}
% ---- custom commands end

%% Document begins here
\begin{document}

%
% --- Title page by R. Follath
% comment the next line for EASY export, as the cover page is handled by the system
\input 0_titlePage

% set page counter to '2', as easy will create the cover page and need the PDF for import to start on page 2.
\clearpage
\setcounter{page}{2}

%% TOC
\tableofcontents

%% Abstract
\begin{abstract}
foobar
\end{abstract}

% include body, the bones and flesh so to say
% \documentclass{article}
% \usepackage[utf8]{inputenc}

% \title{Mechanical Design Guide for Motion Applications}
% \author{marco.schneider }
% \date{April 2021}

% \begin{document}

% \maketitle

\section{Introduction}

\title{Device Motion Components, ElNet Motion}
\author{Marco Schneider, Niko Kivel}
\date{April 2021}



\maketitle

\section{Introduction}

Motion Design Guide
Marco Schneider

\section{Grundlagen}

\subsection{Umgebungsbedingung}
ESS Dokument für infos lesen

\subsection{Personensicherheit}


\subsection{Genauigkeit und Präzision}
Anforderungen klären

\subsection{Fahrweg}
Genügend Spielraum für Bremswege (nach Limitswitch)

\subsection{Geschwindigkeiten}
10 mm/s

\subsection{Endanschläge}
Genügend Stabil

\subsection{Motoren}

\subsection{Encoder}

\subsection{Limitswitches}
Immer zwei, jeweils für pos und neg richtung
Müssen überfahren werden könne

\subsection{Kabel und Stecker}

\subsection{Kupplungen}

\subsection{Linearführungen}

\subsection{Spindellagerung}

\subsection{Schleppketten}

\subsection{Berührungsschutz}
IP2x
	
\subsection{Weiteres}


% \end{document}



\end{document}